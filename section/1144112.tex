\section{Jurnal CNN }
\subsection{Sergio Pereira}

\begin{enumerate}
  \item Judul : MRI Brain Tumor Segmentation Using Convolution Neural Networks in MRI Images
   \item Author : Sérgio Pereira, Adriano Pinto, Victor Alves, and Carlos A. Silva
    \item Jurnal : Judul buku= pereira2016brain, Halaman = 1240-1251, Tahun = 2016
     \item Permasalahan = Penggunaan Metode berbasis CNN untuk segmentasi tumor otak

     
\section{Abstrack}
Di antara tumor otak, glioma adalah yang paling umum dan agresif, yang mengarah ke harapan hidup yang sangat singkat di tingkat tertinggi. Dengan demikian, perencanaan perawatan adalah tahap kunci untuk meningkatkan kualitas hidup pasien onkologis. Magnetic resonance imaging (MRI) adalah teknik pencitraan yang banyak digunakan untuk menilai tumor ini, tetapi sejumlah besar data yang dihasilkan oleh MRI mencegah segmentasi manual dalam waktu yang wajar, membatasi penggunaan pengukuran kuantitatif yang tepat dalam praktik klinis. Jadi, diperlukan metode segmentasi otomatis dan andal; Namun, variabilitas spasial dan struktural yang besar di antara tumor otak membuat segmentasi otomatis menjadi masalah yang menantang. Dalam tulisan ini, kami mengusulkan metode segmentasi otomatis berdasarkan Convolutional Neural Networks (CNN), mengeksplorasi 3 3 kernel. Penggunaan kernel kecil memungkinkan perancangan arsitektur yang lebih dalam, selain memiliki efek positif terhadap overfitting, mengingat jumlah bobot yang lebih sedikit dalam jaringan. Kami juga menyelidiki penggunaan normalisasi intensitas sebagai langkah pra-pemrosesan, yang meskipun tidak umum dalam metode segmentasi berbasis CNN, terbukti bersama dengan augmentasi data menjadi sangat efektif untuk segmentasi tumor otak pada gambar MRI. Proposal kami divalidasi dalam database Brain Tumor Segmentation Challenge 2013 (BRATS
2013), mendapatkan secara simultan posisi pertama untuk daerah yang lengkap, inti, dan meningkatkan dalam metrik Koefisien Kesamaan Dadu (0,88, 0,83, 0,77) untuk kumpulan data Tantangan. Juga, itu diperoleh
posisi keseluruhan pertama oleh platform evaluasi online. Kami juga
berpartisipasi dalam Tantangan BRATS 2015 di lokasi dengan menggunakan model yang sama, mendapatkan tempat kedua, dengan Koefisien Kesamaan Dice
metrik 0,78, 0,65, dan 0,75 untuk lengkap, inti, dan peningkatan
daerah masing-masing.
Istilah Indeks — Tumor otak, segmentasi tumor otak, jaringan saraf konvensional, pembelajaran mendalam, glioma, pencitraan resonansi magnetik
\section{Metode yang digunakan}
 Deep Learning berurusan dengan pembelajaran representasi dengan secara otomatis mempelajari hierarki fitur yang semakin kompleks langsung dari data
\subsection{Hasil dari Metode}
secara simultan diperoleh posisi pertama dalam metrik DSC dalam wilayah lengkap, inti, dan peningkat dalam kumpulan data Tantangan. Dibandingkan dengan model generatif terbaik , kami dapat mengurangi waktu komputasi sekitar sepuluh kali lipat. Mengenai database 2015, kami memperoleh posisi kedua di antara dua belas pesaing dalam tantangan di tempat. Kami berpendapat, oleh karena itu, bahwa komponen yang dipelajari memiliki potensi untuk dimasukkan dalam metode berbasis CNN dan bahwa secara keseluruhan metode kami adalah kandidat kuat untuk segmentasi tumor otak menggunakan gambar MRI.

