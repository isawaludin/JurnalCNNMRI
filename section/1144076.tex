\section{CNNMRI}
\begin{itemize}

\item judul
Multi-scale 3D convolutional neural networks for lesion segmentation in brain MRI

\item author
Kamnitsas, Konstantinos and Chen, Liang and Ledig, Christian and Rueckert, Daniel and Glocker, Ben

\item jurnal
jurnal (Ischemic stroke lesion segmentation), volume=13, halaman(46), tahun (2015)

\item url 
http://www.isles-challenge.org/ISLES2015/pdf/20150930_ISLES2015_Proceedings.pdf#page=21

\item masalah

Kami menghadirkan 11-layer deep, jalur ganda, 3D Convolutional Neural Network kami, yang dikembangkan untuk segmentasi lesi otak.
Segmen sistem yang dikembangkan patologi voxel-bijaksana setelah pemrosesan a
patch 3D multi-modal yang sesuai pada berbagai skala. Kami berdemonstrasi
bahwa mungkin untuk melatih CNN 3D yang mendalam dan lebar pada yang kecil
dataset 28 kasus.

\item kontribusi

Jaringan kami memberikan hasil yang menjanjikan pada tugas
segmentasi lesi stroke iskemik, mencapai Dice rata-rata 64%
(66% setelah postprocessing) pada dataset pelatihan ISLES 2015, peringkat di antara entri teratas.

\item metode/solusi
Network Architecture & Data Preprocessing, Augmentation and Postprocessing

\item hasil
Arsitektur kami menunjukkan kinerja yang menjanjikan, dengan kemampuan yang halus
segmentasi. Kesulitan diamati dalam segmentasi lesi ukuran sangat kecil. Pemisahan lesi ke dalam kategori yang berbeda, misalnya
sesuai dengan ukuran mereka, dan perawatan mereka dengan pengklasifikasi yang terpisah dapat menyederhanakan
tugas untuk setiap pelajar dan membantu meringankan masalah.

\item batasan
Convolutional Neural Networks (CNNs) telah selanjutnya diterapkan
berhasil pada berbagai masalah segmentasi biomedis. Paling maju
pendekatan mengandalkan adaptasi CNN 2D untuk memproses volume 3D, dengan kesulitan dilaporkan saat pelatihan CNN 3D dicoba.
Sementara arsitektur jaringan 2D ini mungkin berhasil dalam beberapa masalah,
mereka tidak optimal dalam penggunaan informasi 3D yang tersedia. 3D murni pertama
Ulasan CNN disajikan dalam, di mana ia digunakan untuk segmentasi tumor otak.
\end{itemize}